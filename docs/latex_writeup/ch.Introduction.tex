The general class of unfolding methods is amongst the physicist's toolbox as a powerful 
means to connect an experiment's observable variables with true physical quantities.
Typically a matrix can be built to encompass the effects of the measurement process
on a simulated `true' distribution and the manifestation of said distribution as an 
experimenter's desired observable. With this response matrix, a distribution of 
the observable in an experiment can be \textbf{unfolded}, providing an estimate of the 
true parent distribution.

A variety of unfolding methods exist, each with its respective strengths and weaknesses.
For example, the simplest method is the matrix inversion unfolding, which for a well 
populated, highly linear response matrix can be both efficient and precise.
However, even with relatively small off-diagonal elements, this method can be unfavorable,
as the matrix may be singular or may introduce wildly fluctuating results due to limited 
statistics. There exist methods to quell such issues, though these require the tuning
of various parameters which typically have no physical connection to the experiment at hand.

Here we discuss D'Agostini's Bayesian unfolding technique presented in \cite{agostini}, a manifestly 
\textbf{inferential} method.
%, and its relation to the cosmic-ray energy reconstruction for the 
%HAWC experiment. 
Starting from Bayes' theorem, an iterative unfolding procedure is developed, 
which then can be implemented without too much difficulty for the typical experimenter.
This document has been adapted from Chapter 7 and Appendix B of \cite{zhampel-thesis}.
